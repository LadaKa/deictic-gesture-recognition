\chapter{Gesture Based Robot Control}

\section{Gesture based "Pick And Place"}
Two gestures are needed to execute the "Pick And Place" task successfully: a pointing gesture to indicate the position and another gesture to confirm it.\par
During the selection of the object and the target location, the user should face the camera, with all objects lying on the floor between him and the camera. The scene is displayed on the rViz screen. Once the objects are detected by the vision system, their images are marked in blue.\par
The user can then initiate the selection of an object by pointing at it with his right hand. To confirm the gesture, the user raises his left hand while still pointing at the object.\par
The object closest to the intersection of the pointing ray and the floor is selected. Its image is marked in red.
The target location of the selected object can then be determined. The user determines the location in the same way as before: by pointing to the location and raising his hand.\par
The target location is on the floor and  has to be selected inside the safety frame that is shown in the rViz. The frame represents a space that is safe for the robot to move around, there are no obstacles except for the detected objects.\par
Once the target location is selected, it is marked in red in rViz, gesture detection is completed and the resulting data is sent to the robot.\par

\section{Gestures}

\subsection{Pointing gesture}
The user can choose from three types of pointing gestures. Each type is represented by a pair of joints that determine the pointing ray. The first joint in the pair is the origin and the second determines the direction of the ray:\par
\begin{itemize}
	\item Shoulder, wrist (default option).
    \item Elbow, wrist.
    \item Head, hand.
\end{itemize}

The pointing gesture indicates the point where the pointing ray intersects the floor and allows selection of the object or target location.\par
There is an option to show or hide the corresponding pointing ray. If the ray is visible, it is displayed from its origin to the intersection with the floor.\par
The pointing gesture has to be performed with the right hand and the first joint has to be positioned higher than the second. These constraints help to reduce the number of falsely detected gestures.\par

\subsection{Raising hand gesture}
The hand gesture consists of lifting the hand. It has to be performed with the left arm and the hand  has to be raised above the head.\par
When pointing with the right arm, the user confirms the pointing gesture by raising the left hand. If no pointing gesture is performed in the moment, the raising gesture is ignored.\par


