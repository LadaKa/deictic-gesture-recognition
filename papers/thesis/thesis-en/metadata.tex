%%% Please fill in basic information on your thesis, which will be automatically
%%% inserted at the right places.

% Type of your thesis:
%	"bc" for Bachelor's
%	"mgr" for Master's
%	"phd" for PhD
%	"rig" for rigorosum
\def\ThesisType{mgr}

% Language of your study programme:
%	"cs" for Czech
%	"en" for English
\def\StudyLanguage{cs}

% Thesis title in English (exactly as in the official assignment)
% (Note: \xxx is a "ToDo label" which makes the unfilled visible. Remove it.)
\def\ThesisTitle{{Manipulating Objects through Deictic Gesture Recognition}}

% Author of the thesis (you)
\def\ThesisAuthor{{Lada Kudláčková}}

% Year when the thesis is submitted
\def\YearSubmitted{{2024}}

% Name of the department or institute, where the work was officially assigned
% (according to the Organizational Structure of MFF UK in English,
% see https://www.mff.cuni.cz/en/faculty/organizational-structure,
% or a full name of a department outside MFF)
\def\Department{{Department of Theoretical Computer Science and Mathematical Logic}}

% Is it a department (katedra), or an institute (ústav)?
\def\DeptType{{Department}}

% Thesis supervisor: name, surname and titles
\def\Supervisor{{RNDr. David Obdržálek, Ph.D.}}

% Supervisor's department (again according to Organizational structure of MFF)
\def\SupervisorsDepartment{{Department of Theoretical Computer Science and Mathematical Logic}}

% Study programme (does not apply to rigorosum theses)
\def\StudyProgramme{{Computer Science}}

% An optional dedication: you can thank whomever you wish (your supervisor,
% consultant, who provided you with tea and pizza, etc.)
\def\Dedication{%
\xxx{Dedication.}
}

% Abstract (recommended length around 80-200 words; this is not a copy of your thesis assignment!)
\def\Abstract{%
{A gesture based control for the Pick And Place task was designed and implemented.
Several types of gestures were explored.
A raised hand gesture was selected for confirmation and a pointing gesture to specify the object and its target position.
Experiments were conducted to measure the accuracy of three different types of pointing gestures.
Gesture recognition is based on image processing and uses data from a depth camera.
Tests of the control in a real environment were performed with a mobile robotic manipulator.}
}

% 3 to 5 keywords (recommended) separated by \sep
% Keywords are useful for indexing and searching for the theses by topic.
\def\ThesisKeywords{%
{gesture recognition\sep object manipulation\sep autonomous control}
}

% If any of your metadata strings contains TeX macros, you need to provide
% a plain-text version for use in XMP metadata embedded in the output PDF file.
% If you are not sure, check the generated thesis.xmpdata file.
\def\ThesisAuthorXMP{\ThesisAuthor}
\def\ThesisTitleXMP{\ThesisTitle}
\def\ThesisKeywordsXMP{\ThesisKeywords}
\def\AbstractXMP{\Abstract}

% If your abstracts are long and do not fit in the infopage, you can make the
% fonts a bit smaller by this setting. (Also, you should try to compress your abstract more.)
\def\InfoPageFont{}
%\def\InfoPageFont{\small}  % uncomment to decrease font size

% If you are studing in a Czech programme, you also need to provide metadata in Czech:
% (in English programmes, this is not used anywhere)

\def\ThesisTitleCS{{Manipulace s objekty pomocí rozpoznávání ukazovacích gest}}
\def\DepartmentCS{{Katedra teoretické informatiky a matematické logiky}}
\def\DeptTypeCS{{Katedra}}
\def\SupervisorsDepartmentCS{{Katedra teoretické informatiky a matematické logiky}}
\def\StudyProgrammeCS{{studijní program}}

\def\ThesisKeywordsCS{%
{rozpoznání gest\sep manipulace s objekty\sep autonomní řízení}
}

\def\AbstractCS{%
{Pro úlohu Pick And Place bylo navrženo a implementováno ovládání založené na deiktických gestech.
Bylo prozkoumáno několik typů gest. Pro potvrzování bylo vybráno gesto zvednuté ruky a pro specifikaci objektu a jeho cílové polohy  gesto ukazování.
Byly provedeny experimenty měřící přesnost tří různých typů ukazovacích gest.
Rozpoznávání gest je založeno na zpracování obrazu a používá data hloubkové kamery.
Testy ovládání v reálném prostředí byly provedeny s mobilním robotickým manipulátorem.}
}
