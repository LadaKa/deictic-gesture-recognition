\chapter{The state of the art}

\section{History of Gesture Based Control}
\subsection{Computer Vision}
The first digital image scanner was built in 1957 by Russell A. Kirsch. The device optically scanned a scene and converted the scan into an image, which was represented by pixels.\par
In the 1960s, David Hubel and Torsten Wiesel conducted experiments with cats that provided insight into image processing in the brain. They showed cats various simple visual stimuli while recording the electrical activity of cells in the visual cortex.\par
This revealed how the brain builds complex visual representations from simple elements and how individual neurons are involved in image processing, e.g. neurons responsible for edge detection were discovered.\par
In 1966 was founded the Summer Vision Project on MIT, which focused on machine vision and pattern recognition. The goal was to implement a visual system that would solve tasks such as distinguishing between foreground and background or extracting distinct objects.\par
Edge detection was implemented in 1987 using the gradient calculation.\par
The "Eigenfaces" face recognition algorithm was developed in 1991, the Scale-Invariant Feature Transform (SIFT) algorithm for local features detection in 1999, and the Viola-Jones face detection algorithm in 2001.\par
In 2005, the Histogram of Oriented Gradients (HOG) algorithm was introduced that enabled efficient body recognition and object detection.\par
Deep learning has become the dominant computer vision method in the following years, especially after the success of the ImageNet Challenge in 2012.\par

\subsection{Gesture Recognition}
In the 1960s, the first touch screens and pointing devices were developed. In 1963, Sketchpad was written, a computer program that allowed interaction with objects on the screen using a light pen to capture motion.\par
In 1977, Sayre's Glove was introduced, which used a light sensor and a flexible light source tube to determine the position of the fingers. Later, various sensors such as accelerometers or magnetic sensors were used for gesture recognition using gloves.\par
Image processing tools enabled vision-based hand gesture recognition using images of gestures performed with a color marked glove.\par
Body recognition methods included wearable sensors such as electromyographic (EMG) sensors attached to the arm or suit with an IMU and barometer.\par
Sensors such as radar enabled body recognition at a distance.\par

Significant was the development of depth cameras that use time-of-flight or structured light-based technologies, such as the Microsoft Kinect device launched in 2010, which enabled real-time body and gesture recognition.\par

\section{Localization and navigation with deictic gestures}

These are some (not all) examples of what I want to mention here:\\
 
Deictic gestures for multi-robot systems \\

Paper: \\
B. Gromov, L. M. Gambardella and G. A. Di Caro, "Wearable multi-modal interface for human multi-robot interaction," 2016 IEEE International Symposium on Safety, Security, and Rescue Robotics (SSRR), Lausanne, Switzerland, 2016, pp. 240-245, doi: 10.1109/SSRR.2016.7784305.\\

Use of the pointing gesture for localization \\

Paper: \\
B. Gromov, L. Gambardella, and A. Giusti. Robot Identification and Localization with Pointing Gestures. IEEE/RSJ International Conference on Intelligent Robots and Systems (IROS), 2018, pp. 3921–3928 https://people.idsia.ch/~gromov/repository/gromov2018robot.pdf \\

3D Motion planning with pointing gestures \\

Paper: \\
B. Gromov, J. Guzzi, L. Gambardella, and A. Giusti. Intuitive 3D Control of a Quadrotor in User Proximity with Pointing Gestures. IEEE International Conference on Robotics and Automation (ICRA), 2020 https://people.idsia.ch/~gromov/repository/gromov2020intuitive.pdf

\section{Interpretation of gestures}
Papers:\\

Chaudhary, A (2018). Robust Hand Gesture Recognition for Robotic Hand Control. Springer. ISBN 978-981-10-4798-5 https://doi.org/10.1007/978-981-10-4798-5\\

Alikhani, M., Khalid, B., Shome, R., Mitash, C., Bekris, K.E., Stone, M. (2020). That and There: Judging the Intent of Pointing Actions with Robotic Arms. AAAI.https://ojs.aaai.org//index.php/AAAI/article/view/6601\\

\section{Object detection with pointing gestures and speech recognition}
Li-Heng Lin, Yuchen Cui, Yilun Hao, Fei Xia, Dorsa Sadigh (2023). Gesture-Informed Robot Assistance via Foundation Models. https://arxiv.org/abs/2309.02721 \\

A. Ekrekli, A. Angleraud, G. Sharma, R. Pieters (2023). Co-speech gestures for human-robot collaboration. https://arxiv.org/abs/2311.18285 \\



% 

% 

% 
%

% paper about virtual environment

% some new papers from 2024?